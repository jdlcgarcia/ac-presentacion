\documentclass{beamer}

\usepackage[utf8]{inputenc}
\usepackage[spanish,es-noshorthands]{babel}
\usepackage[utf8]{inputenc}
\usepackage{amssymb,amsthm,amsmath,tikz,graphicx,caption,subcaption}

\usetikzlibrary{intersections, arrows.meta, automata, er, calc, backgrounds, mindmap, folding, patterns, decorations.markings, fit,shadings, matrix, positioning, arrows, through}
\usetheme{PaloAlto}
\usecolortheme{seahorse}


%Information to be included in the title page:
\title{Algoritmos de compresión en medios de preservación y difusión cultural}
\author{Juan de la Cruz García García}
\institute{Actualización Científica - Master en Matemáticas - Universidad de Cádiz}
\date{2021.03.14}

\begin{document}

\frame{\titlepage}

\begin{frame}
\frametitle{Table of Contents}
\tableofcontents
\end{frame}

\section{Introducción}

\begin{frame}
\frametitle{¿De qué hablaremos hoy?}

\begin{tikzpicture}
	[
	node distance=0.5cm and 1cm,
        destacado/.style={
            rectangle, 
            rounded corners,
            draw=red!50,
            very thick,
            fill=red!5,
            align=center,
            text width=2.5cm
        }, 
        explicacion/.style={
	    rectangle, 
	    rounded corners,
            align=left,
            draw=violet!50,
            fill=violet!10,
            very thick,
            text width=6.25cm
        }, 
        flechaAccion/.style={
            ->,
            thick
        }
    ]
    \node(algoritmos) [destacado] {Algoritmos de compresión};
    \node(medios) [destacado, below=of algoritmos] {Medios};
    \node(preservar) [destacado, below=0.75cm of medios] {Preservar};
    \node(difundir) [destacado, below=1.25cm of preservar] {Difusión};
    \node(cultura) [destacado, below=1cm of difundir] {Cultura};

    \node(explicacionAlgoritmos) [explicacion, right=of algoritmos] {Matemáticas};
    \node(explicacionMedios) [explicacion, right=of medios] {Formatos, objetos, vías};
    \node(explicacionPreservar) [explicacion, right=of preservar] {Proteger, resguardar anticipadamente a alguien o algo, de algún daño o peligro};
    \node(explicacionDifundir) [explicacion, right=of difundir] {Propagar o divulgar conocimientos, noticias, actitudes, costumbres, modas, etc};
    \node(explicacionCultura) [explicacion, right=of cultura] {Música, Literatura, Cine, Pintura, Escultura...};

    \draw[flechaAccion] (algoritmos)--(explicacionAlgoritmos);
    \draw[flechaAccion] (medios)--(explicacionMedios);
    \draw[flechaAccion] (preservar)--(explicacionPreservar);
    \draw[flechaAccion] (difundir)--(explicacionDifundir);
    \draw[flechaAccion] (cultura)--(explicacionCultura);
\end{tikzpicture}

%\begin{columns}

%\column{0.5\textwidth}
%This is a text in first column.
%$$E=mc^2$$
%\begin{itemize}
%\item First item
%\item Second item
%\end{itemize}

%\column{0.5\textwidth}
%This text will be in the second column
%and on a second tought this is a nice looking
%layout in some cases.
%\end{columns}
\end{frame}

\section{Formatos Analógicos y digitales}
\section{Algoritmos sin compresión}
\section{Algoritmos de compresión con pérdidas}
\section{Algoritmos de compresión sin pérdidas}
\section{Conveniencia para cada uso}



\end{document}
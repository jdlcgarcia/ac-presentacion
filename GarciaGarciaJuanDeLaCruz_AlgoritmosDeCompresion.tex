\documentclass{beamer}

\usepackage[utf8]{inputenc}
\usepackage[spanish,es-noshorthands]{babel}
\usepackage[utf8]{inputenc}
\usepackage{amssymb,amsthm,amsmath,tikz,graphicx,caption,subcaption}

\usetikzlibrary{intersections, arrows.meta, automata, er, calc, backgrounds, mindmap, folding, patterns, decorations.markings, fit,shadings, matrix, positioning, arrows, through}
\usetheme{PaloAlto}
\usecolortheme{seahorse}

%Information to be included in the title page:
\title{Algoritmos de compresión en medios de preservación y difusión cultural}
\author{Juan de la Cruz García García}
\institute{Actualización Científica - Master en Matemáticas - Universidad de Cádiz}
\date{2021.03.14}

\begin{document}

\frame{\titlepage}

\begin{frame}
\frametitle{Table of Contents}
\tableofcontents
\end{frame}

\section{Introducción}

\begin{frame}
\frametitle{¿De qué hablaremos hoy?}

\begin{tikzpicture}
	[
	node distance=0.5cm and 1cm,
        destacado/.style={
            rectangle, 
            rounded corners,
            draw=red!50,
            very thick,
            fill=red!5,
            align=center,
            text width=2.5cm
        }, 
        explicacion/.style={
	    rectangle, 
	    rounded corners,
            align=left,
            draw=violet!50,
            fill=violet!10,
            very thick,
            text width=6.25cm
        }, 
        flechaAccion/.style={
            ->,
            thick
        }
    ]
    \node(algoritmos) [destacado] {Algoritmos de compresión};
    \node(medios) [destacado, below=of algoritmos] {Medios};
    \node(preservar) [destacado, below=0.75cm of medios] {Preservar};
    \node(difundir) [destacado, below=1.25cm of preservar] {Difusión};
    \node(cultura) [destacado, below=1cm of difundir] {Cultura};

    \node(explicacionAlgoritmos) [explicacion, right=of algoritmos] {Matemáticas};
    \node(explicacionMedios) [explicacion, right=of medios] {Formatos, objetos, vías};
    \node(explicacionPreservar) [explicacion, right=of preservar] {Proteger, resguardar anticipadamente a alguien o algo, de algún daño o peligro};
    \node(explicacionDifundir) [explicacion, right=of difundir] {Propagar o divulgar conocimientos, noticias, actitudes, costumbres, modas, etc};
    \node(explicacionCultura) [explicacion, right=of cultura] {Música, Literatura, Cine, Pintura, Escultura...};

    \draw[flechaAccion] (algoritmos)--(explicacionAlgoritmos);
    \draw[flechaAccion] (medios)--(explicacionMedios);
    \draw[flechaAccion] (preservar)--(explicacionPreservar);
    \draw[flechaAccion] (difundir)--(explicacionDifundir);
    \draw[flechaAccion] (cultura)--(explicacionCultura);
\end{tikzpicture}
\end{frame}

\begin{frame}
    \frametitle{Algoritmos de compresión}
    \centering
		\setbeamercovered{dynamic}
		\begin{tikzpicture}[scale=0.75,transform shape]
		\path[mindmap,concept color=green!60!black,text=white]
		node[concept] {Algoritmos de compresión}
		[clockwise from=-45]
		    child[concept color=teal!30!white] {
			node[concept] {Con pérdida}
			[clockwise from=45]
			child { node[concept] {Transforma\-da de Coseno Discreta (DCT)} }
			child { node[concept] {MCDT} }
			}  
			child[concept color=purple!20!white]{
			node[concept] {Sin pérdida}
			[clockwise from=-10]
			child {node[concept]{LZ77}}
			child {node[concept]{LZR}}
			child {node[concept]{LZSS}}
			child {node[concept]{LZMA}}
            child {node[concept]{bzip2}}
		};
		
		\end{tikzpicture}
\end{frame}

\begin{frame}
    \frametitle{Algoritmos, Formatos y Estándares}
    \centering
		\setbeamercovered{dynamic}
		\begin{tikzpicture}
            [
            node distance=0.5cm and 1cm,
                circulo/.style={
                    circle, 
                    draw=red!50,
                    very thick,
                    fill=red!5,
                    align=center,
                    text width=2.5cm
                }, 
                explicacion/.style={
                rectangle, 
                rounded corners,
                    align=left,
                    draw=violet!50,
                    fill=violet!10,
                    very thick,
                    text width=6.25cm
                }, 
                flechaAccion/.style={
                    ->,
                    thick
                }
            ]
            
            \draw[black,thick,fill=teal!10] (0,0) circle (3.5cm);
            \node[] at (0,3){Estandar};
            \draw[black,thick,fill=purple!10] (0,0) circle (2.5cm);
            \node[] at (0,2){Formato};
            \draw[black,thick,fill=green!10] (0,0) circle (1.5cm);
            \node[] at (0,0){Algoritmo};
            %\node(algoritmos) [circle, draw=teal!70,very thick,fill=purple!5,align=center,text width=4.5cm, label=south:Formato] {Formato};
            %\node(algoritmos) [circle, draw=teal!50,very thick,fill=purple!5,align=center,text width=2.5cm] {Algoritmo};
            
		
		\end{tikzpicture}
\end{frame}

\begin{frame}
    \frametitle{Nomenclatura casual}
    \centering
		\setbeamercovered{dynamic}
		\begin{tikzpicture}[scale=0.75,transform shape]
		\path[mindmap,concept color=green!60!black,text=white]
		node[concept] {Formatos de compresión}
		[clockwise from=-45]
		    child[concept color=teal!30!white] {
			node[concept] {Con pérdida}
			[clockwise from=90]
			child { node[concept] {JPG} }
			child { node[concept] {WebP} }
            child { node[concept] {MP3} }
            child { node[concept] {H.264} }
			}  
			child[concept color=purple!20!white]{
			node[concept] {Sin pérdida}
			[clockwise from=-10]
			child {node[concept]{PNG}}
			child {node[concept]{BMP}}
			child {node[concept]{AVI}}
			child {node[concept]{ALAC}}
            child {node[concept]{WAV}}
		};
		
		\end{tikzpicture}
\end{frame}
%\begin{columns}

%\column{0.5\textwidth}
%This is a text in first column.
%$$E=mc^2$$
%\begin{itemize}
%\item First item
%\item Second item
%\end{itemize}

%\column{0.5\textwidth}
%This text will be in the second column
%and on a second tought this is a nice looking
%layout in some cases.
%\end{columns}

\section{Formatos Analógicos y digitales}
\section{Algoritmos sin compresión}
\section{Algoritmos de compresión con pérdidas}
\section{Algoritmos de compresión sin pérdidas}
\section{Conveniencia para cada uso}



\end{document}